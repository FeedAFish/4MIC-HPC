% Options for packages loaded elsewhere
\PassOptionsToPackage{unicode}{hyperref}
\PassOptionsToPackage{hyphens}{url}
%
\documentclass[
  12pt,
  xcolor = usenames,dvipsnames]{article}
\usepackage{amsmath,amssymb}
\usepackage{lmodern}
\usepackage{iftex}
\ifPDFTeX
  \usepackage[T1]{fontenc}
  \usepackage[utf8]{inputenc}
  \usepackage{textcomp} % provide euro and other symbols
\else % if luatex or xetex
  \usepackage{unicode-math}
  \defaultfontfeatures{Scale=MatchLowercase}
  \defaultfontfeatures[\rmfamily]{Ligatures=TeX,Scale=1}
\fi
% Use upquote if available, for straight quotes in verbatim environments
\IfFileExists{upquote.sty}{\usepackage{upquote}}{}
\IfFileExists{microtype.sty}{% use microtype if available
  \usepackage[]{microtype}
  \UseMicrotypeSet[protrusion]{basicmath} % disable protrusion for tt fonts
}{}
\makeatletter
\@ifundefined{KOMAClassName}{% if non-KOMA class
  \IfFileExists{parskip.sty}{%
    \usepackage{parskip}
  }{% else
    \setlength{\parindent}{0pt}
    \setlength{\parskip}{6pt plus 2pt minus 1pt}}
}{% if KOMA class
  \KOMAoptions{parskip=half}}
\makeatother
\usepackage{xcolor}
\IfFileExists{xurl.sty}{\usepackage{xurl}}{} % add URL line breaks if available
\IfFileExists{bookmark.sty}{\usepackage{bookmark}}{\usepackage{hyperref}}
\hypersetup{
  hidelinks,
  pdfcreator={LaTeX via pandoc}}
\urlstyle{same} % disable monospaced font for URLs
\usepackage[margin=1in]{geometry}
\usepackage{listings}
\newcommand{\passthrough}[1]{#1}
\lstset{defaultdialect=[5.3]Lua}
\lstset{defaultdialect=[x86masm]Assembler}
\usepackage{longtable,booktabs,array}
\usepackage{calc} % for calculating minipage widths
% Correct order of tables after \paragraph or \subparagraph
\usepackage{etoolbox}
\makeatletter
\patchcmd\longtable{\par}{\if@noskipsec\mbox{}\fi\par}{}{}
\makeatother
% Allow footnotes in longtable head/foot
\IfFileExists{footnotehyper.sty}{\usepackage{footnotehyper}}{\usepackage{footnote}}
\makesavenoteenv{longtable}
\usepackage{graphicx}
\makeatletter
\def\maxwidth{\ifdim\Gin@nat@width>\linewidth\linewidth\else\Gin@nat@width\fi}
\def\maxheight{\ifdim\Gin@nat@height>\textheight\textheight\else\Gin@nat@height\fi}
\makeatother
% Scale images if necessary, so that they will not overflow the page
% margins by default, and it is still possible to overwrite the defaults
% using explicit options in \includegraphics[width, height, ...]{}
\setkeys{Gin}{width=\maxwidth,height=\maxheight,keepaspectratio}
% Set default figure placement to htbp
\makeatletter
\def\fps@figure{htbp}
\makeatother
\setlength{\emergencystretch}{3em} % prevent overfull lines
\providecommand{\tightlist}{%
  \setlength{\itemsep}{0pt}\setlength{\parskip}{0pt}}
\setcounter{secnumdepth}{5}
\usepackage{setspace}
\usepackage{float}
\usepackage{fontspec}
\usepackage{subfig}
\setmonofont{JetBrains Mono}[Contextuals=Alternate]
\floatplacement{figure}{H}
\lstset{
  language=C,
  basicstyle={\linespread{0.8}\ttfamily},
  stepnumber=1,
  numbersep=5pt,
  backgroundcolor=\color{cyan!5},
  showspaces=false,
  showstringspaces=false,
  showtabs=false,
  frame=single,
  rulecolor=\color{black},
  tabsize=2,
  captionpos=b,
  breaklines=true,
  breakatwhitespace=false,
  keywordstyle=\color{RoyalBlue},
  commentstyle=\color{Green},
  stringstyle=\color{Orange},
}
\makeatletter
\renewcommand\paragraph{\@startsection{paragraph}{4}{\z@}%
  {-2.5ex\@plus -1ex \@minus -.25ex}%
  {1.25ex \@plus .25ex}%
  {\normalfont\normalsize\bfseries}}
\makeatother
\setcounter{secnumdepth}{4}
\hypersetup{
  colorlinks = true,
}
\usepackage{booktabs}
\usepackage{longtable}
\usepackage{array}
\usepackage{multirow}
\usepackage{wrapfig}
\usepackage{float}
\usepackage{colortbl}
\usepackage{pdflscape}
\usepackage{tabu}
\usepackage{threeparttable}
\usepackage{threeparttablex}
\usepackage[normalem]{ulem}
\usepackage{makecell}
\usepackage{xcolor}
\ifLuaTeX
  \usepackage{selnolig}  % disable illegal ligatures
\fi

\author{}
\date{\vspace{-2.5em}}

\begin{document}

\onehalfspacing

\pagenumbering{gobble}

%\begin{titlepage}
\vspace*{\fill}
\begin{center}
  \LARGE{\textbf{Optimization and OpenMP parallelization of the dense matrix-matrix product computation}}\\
  \Large{\textbf{HPC 4GMM 2021/2022}}\\
  \vspace*{1\baselineskip}
  \Large{\textbf{Members}}\\
  PHAM Tuan Kiet\\
  VO Van Nghia\\
  \vfill % equivalent to \vspace{\fill}
  \vspace*{\fill}
  \Large{\textbf{Date}}\\
  12 Dec, 2021
\end{center}
% \end{titlepage}

\newpage

\newpage
\pagenumbering{roman}
\tableofcontents
\addcontentsline{toc}{section}{\contentsname}

\listoffigures

\newpage
\pagenumbering{arabic}

\hypertarget{techniques}{%
\section{Techniques}\label{techniques}}

\hypertarget{native-dot}{%
\subsection{Native dot}\label{native-dot}}

We first mention here the original \passthrough{\lstinline!native\_dot!} function. This function serves as an anchor (or base case) for performance comparision as well as for making sure we have the right result when using other techniques.

\begin{lstlisting}[language=C]
for (i = 0; i < M; i++)
  for (j = 0; j < N; j++)
    for (k = 0; k < K; k++) C[i + ldc * j] += A[i + lda * k] * B[k + ldb * j];
\end{lstlisting}

Below is the output of \passthrough{\lstinline!native\_dot!} for \passthrough{\lstinline!M = 1!}, \passthrough{\lstinline!K = 2!}, \passthrough{\lstinline!N = 2!}:

\begin{lstlisting}
## 
## Parallel execution with a maximum of 4 threads callable
## 
## Scheduling static with chunk = 0
## 
## ( 1.00  1.50 )
## 
## ( 1.00  1.50 )
## ( 1.50  2.00 )
## 
## Frobenius Norm   = 5.550901
## Total time naive = 0.000000
## Gflops           = inf
## 
## ( 3.25  4.50 )
\end{lstlisting}

As
\[
\begin{pmatrix}
1 && 1,5
\end{pmatrix}\begin{pmatrix}
1 && 1,5\\
1,5 && 2
\end{pmatrix}=\begin{pmatrix}
3,25 && 4,5
\end{pmatrix}
\]
The result of this function is correct. We could move on to the next technique.

\hypertarget{spatial-locality}{%
\subsection{Spatial locality}\label{spatial-locality}}

Spatial locality refers to the following scenario: if a particular storage location is referenced at a particular time, then it is likely that nearby memory locations will be referenced in the near future. In order to take advantages of this property, we notice that:

\begin{itemize}
\tightlist
\item
  In memory, \passthrough{\lstinline!A!}, \passthrough{\lstinline!B!}, \passthrough{\lstinline!C!} are stored in contiguous memory block.
\item
  When using the index order \passthrough{\lstinline!i!}, \passthrough{\lstinline!j!}, \passthrough{\lstinline!k!}, we access \passthrough{\lstinline!B!} consecutively (as we access \passthrough{\lstinline!B!} by \passthrough{\lstinline!B[k + ldb * j]!}), but not \passthrough{\lstinline!A!} and \passthrough{\lstinline!C!}.
\item
  Data from \passthrough{\lstinline!A!}, \passthrough{\lstinline!B!}, \passthrough{\lstinline!C!} are loaded in a memory block consisting of severals consecutive elements to cache. Thus, we could make use of spatial locality when reading data continously.
\end{itemize}

From 3 points above, we decide to switch the index order to \passthrough{\lstinline!k!}, \passthrough{\lstinline!j!}, \passthrough{\lstinline!i!}. Now we see that both reading and writing operations on \passthrough{\lstinline!C!} are in cache, this brings us a critical gain in performance. In addition, reading operations on \passthrough{\lstinline!A!} are in cache too but those on \passthrough{\lstinline!B!} are not.

\begin{lstlisting}[language=C]
for (k = 0; k < K; k++)
  for (j = 0; j < N; j++)
    for (i = 0; i < M; i++) C[i + ldc * j] += A[i + lda * k] * B[k + ldb * j];
\end{lstlisting}

For comparision, we have a table below with small \passthrough{\lstinline!M!}, \passthrough{\lstinline!K!}, \passthrough{\lstinline!N!} (\passthrough{\lstinline!OMP!} indicates if we enable \passthrough{\lstinline!Open MP!} or not).

\begin{longtable}[t]{lrrrrrrl}
\toprule
Technique & Time & Norm & Gflops & M & K & N & OMP\\
\midrule
\cellcolor{gray!6}{Naive} & \cellcolor{gray!6}{0} & \cellcolor{gray!6}{3.461352} & \cellcolor{gray!6}{Inf} & \cellcolor{gray!6}{4} & \cellcolor{gray!6}{8} & \cellcolor{gray!6}{4} & \cellcolor{gray!6}{FALSE}\\
Saxpy & 0 & 3.461352 & Inf & 4 & 8 & 4 & FALSE\\
\bottomrule
\end{longtable}

We have the frobenius norm of both techniques are 3,461352, which indicate we have the right computation result. In addition, calculating time is already significantly small (\(\approx\) 0 second in both methods) and the difference between these two can therefore be ommited.

However, if we set \passthrough{\lstinline!M!}, \passthrough{\lstinline!K!}, \passthrough{\lstinline!N!} to 2048. There will be a huge performance gain as in the table shown below.

\begin{longtable}[t]{lrrrrrrl}
\toprule
Technique & Time & Norm & Gflops & M & K & N & OMP\\
\midrule
\cellcolor{gray!6}{Naive} & \cellcolor{gray!6}{82.764972} & \cellcolor{gray!6}{2.323362} & \cellcolor{gray!6}{0.207574} & \cellcolor{gray!6}{2048} & \cellcolor{gray!6}{2048} & \cellcolor{gray!6}{2048} & \cellcolor{gray!6}{FALSE}\\
Saxpy & 4.022001 & 2.323362 & 4.271473 & 2048 & 2048 & 2048 & FALSE\\
\bottomrule
\end{longtable}

Here, the \passthrough{\lstinline!native\_dot!} function is approximately 21 times slower than the \passthrough{\lstinline!saxpy\_dot!} function.

\end{document}
